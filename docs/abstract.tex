\section*{Zusammenfassung}

Diese Arbeit wird sich mit verschiedenen 
Optimierungsmethoden neuronaler Netze beschäftigen
und Sie auf Basis von Beispiel Datensätzen evaluieren,
wie dem Boston House Price Datensatz.
Hierbei wird eine Metrik definiert um die Ergebnis der einzelnen 
Optimierungsmethode zu vergleichen. 

Eine Optimierungsmethode ist eine Möglichkeit
die Konvergenz der Fehlerfunktion $J(\theta)$ des neuronalen Netzes
beim Lernen zu verbessern. 
 
Hierbei wird auf den Lern Prozess des Neuronalen
Netzes eingegangen. Besonderen Fokus wird der 'Gradient Descent',
zu Deutsch Gradienten abstieg, einnehmen, da dies die Grundlage
des Lernens darstellt. Dieser sucht im mehrdimensionalen Raum
die Minima der nichtlinearen Fehlerfunktion und
es gibt verschiedene Möglichkeiten diese Suche zu verbessern.   
Nach der theoretischen Aufarbeitung,
werden wir ein paar Eigenschaften über
diese Optimierungsmethoden annehmen
und diese anhand der Test Daten überprüfen.


\section*{Abstract}

This work will focus on explaning the different optimization methods of neural networks and evaluating them
on example datasets like the boston house price data.
We will define a metric to be able to evaluate the performance 
of the different optimizer.

An optimization method is a way to improve the performance of the error 
function $J(\theta)$ of the neural network. 


Furthermore this work will give a detailed explanation of the learning process of Neural Networks
especially focusing on the Gradient Descent, which is the foundation of learning in neural networks. This algorithm aims to find local minima in the Hyperplane of the non-linear error function and there are multiple ways to improve its search. 
After the Theory, we will assume some properties about those optimization methods and test those assumptions
by evaluating the metrics of these neural networks.

